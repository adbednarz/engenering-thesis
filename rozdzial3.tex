\chapter{Opis technologii}
\thispagestyle{chapterBeginStyle}

W rozdziale tym wskazano jakie podejście technologiczne zostało użyte wraz z omówieniem i zaargumentowaniem powodu takiego wyboru. Przedstawiono również jakie konkretne aspekty tych technologii zostały wykorzystane w projekcie.

\section{Flutter}

Do implementacji serwisu po stronie klienta użyto technologii Flutter. Jest to zestaw narzędzi pozwalający tworzyć natywne, wieloplatformowe aplikacje mobilne, komputerowe oraz internetowe. Flutter stworzony jest przez firmę Google, a jego pierwsza stabilna wersja ukazała się pod koniec 2019 roku. Mimo stosunkowo młodej technologii Google przyczynia się do jej dynamicznego rozwoju i zdobywania popularności wśród programistów. Aplikacje we Flutterze pisze się w języku Dart. Jest to zorientowany obiektowo, statycznie typowany, wysokopoziomowy język programowania.
Składnia języka Dart wzorowana była na takich językach programowania jak Java czy C\#, aby ułatwić programistom naukę nowego języka.

\subsection{Architektura}

Flutter domyślnie nie oddziela kodu dotyczącego sposobu działania programu od części wizualnej. Chcąc wyodrębnić aspekty wizualne od aspektów logicznych aplikacji, należy użyć wybranego wzorca architektury. Istnieje wiele pakietów pomagających zaimplementować poszczególne wzorce. W tym projekcie zastosowano rekomendowany przez Google pakiet BLoC, który umożliwia implementacje architektury trójwarstwowej z podziałem na warstwę wizualną, logiki i danych, jak również wprowadza system zarządzania stanami w aplikacji.

\begin{figure}[h!]
	\begin{center}
		\includegraphics[width=1\textwidth]{img/cubit.png}
	\end{center}
	\caption{{\color{dgray}Model struktury wzorca projektowego BLoC.}} 
	\label{struktura_BLoC}
\end{figure}  

BLoC (Business Logic Component) - wzorzec projektowy pomagający oddzielić elementy wizualne projektu od części logiki działania programu. Dzieli projekt na trzy główne komponenty UI, cubit, data.

Warstwa UI jest odpowiedzialna za część wizualną aplikacji, cubit jest warstwą zawierającą mechanizmy działania aplikacji oraz pomostem pomiędzy interfejsem użytkownika, a zewnętrznymi danymi, natomiast warstwa data komunikuje się z zewnętrznymi instancjami (serwer, baza danych).\\

Cubit komunikuje się z warstwą danych poprzez funkcje asynchroniczne. Natomiast chcąc wpłynąć na zmianę wyglądu aplikacji korzysta z programowania reaktywnego. Mianowicie wysyła poszczególne stany strumieniem, którego dany element interfejsu aplikacji nasłuchuje. Stan jest to klasa reprezentująca dany stan poszczególnego elementu aplikacji. Relacja w drugą stronę polega na wywoływaniu bezpośrednio funkcji (synchronicznej lub asynchronicznej) na instancji cubita.

\section{Firebase}

Rolę serwera i bazy danych pełni platforma Firebase. Technologia ta, zarządzana również przez Google, określana jest jako Backend-as-a-Service. Firebase dostarcza wiele funkcjonalności wspierające tworzenie projektów w tym między innymi uwierzytelnianie użytkowników, bazę danych NoSQL oraz API odpowiedzialne za komunikację.

\subsection{Firestore}

\subsection{Funkcje Firebase}

\subsection{Emulator}

Korzystając z funkcjonalności funkcji Firebase napotyka się na problem długiego czasu oczekiwania na zaktualizowanie kodu źródłowego programisty w chmurze Firebase (czas trwania około minuty). Oczekiwanie dłuższe niż parę sekund sprawia, że czas tworzenia kodu źródłowego po stronie serwera znacząco się wydłuża. Kolejną sprawą jest fakt, że w trakcie, gdy dany projekt jest już dostępny dla użytkowników, nie jest wskazana jego modyfikacja. Zatem trzeba tworzyć kopie projektu, które przeznaczone są modyfikacji, co w przypadku, gdy nad projektem pracuje wielu programistów, staje się jeszcze bardziej nieefektywne. Ostatnim aspektem są koszty. W momencie programowania i testowania funkcjonalności serwera może dochodzić do generowania dużej ilości zapytań, co z kolei sumuje się do liczby zapytań i ewentualnych płatności. \\

Z powodu wyżej wymienionych problemów wraz z rozwojem platformy Firebase w 2019 roku oficjalnie został zaprezentowany Firebase Local Emulators. Emulator ten pozwala uruchomić wszystkie usługi Firebase lokalnie na własnym komputerze. Dzięki czemu szybko można nanieść zmiany w funkcjach Firebase, nie modyfikując właściwego projektu, a wykonywane zapytania nie są zliczane. \\

Każda usługa udostępniana przez platformę Firebase ma swój własny emulator, który jest uruchamiany na osobnym porcie. Zatem można uruchomić jedynie te emulatory tych funkcjonalności, które są wykorzystywane w projekcie. Ponadto fakt uruchomienia lokalnie funkcji Firebase  umożliwia debugowanie kodu.



