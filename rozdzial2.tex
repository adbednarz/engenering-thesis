\chapter{Projekt serwisu}
\thispagestyle{chapterBeginStyle}

W tym rozdziale przedstawiono szczegółowy projekt systemu w notacji UML uwzględniający wymagania funkcjonalne opisane we wstępie. Do opisu relacji pomiędzy składowymi systemu wykorzystano diagramy UML.

\section{Struktura}

W serwisie wykorzystano architekturę klient-serwer. Istnieje jeden serwer, do którego może podłączać się wiele klientów i odpowiada on za komunikację i zarządzanie danymi.
W komponencie klienta wykorzystano architekturę trójwarstwową. Architektura ta dzieli komponent na trzy osobne części:
\begin{itemize}
	\item warstwa prezentacji,
	\item warstwa biznesowa,
	\item warstwa danych.
\end{itemize}
Warstwa prezentacji jest to interfejs graficzny użytkownika. Jest odpowiedzialna za interakcję z użytkownikiem (wyświetlanie i wprowadzanie danych). \\ \\
Warstwa biznesowa odpowiada za przetwarzanie komunikatów od użytkownika lub ze strony serwera. Tutaj zawarta jest wszelka logika aplikacji. Przetworzone dane są przekazywane do warstwy prezentacji i/lub warstwy danych. Warstwa ta jest łącznikiem pomiędzy warstwą prezentacji, a warstwą danych. \\ \\
Warstwa danych jest dostępem do danych. Obsługuje połączenie aplikacji z zewnętrznym obiektem dostarczającym dane (baza danych, serwer).

\section{Przypadki użycia i scenariusze}



\section{Diagramy aktywności}


\section{Diagramy stanów}




