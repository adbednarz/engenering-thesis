%Korekta ALD - nienumerowany wstęp
%\chapter{Wstęp}
\addcontentsline{toc}{chapter}{Wstęp}
\chapter*{Wstęp}

\thispagestyle{chapterBeginStyle}

Celem pracy jest zaprojektowanie i implementacja serwisu informatycznego, który umożliwia przeprowadzanie gier Rummikub.
\\

Założenia funkcjonalne serwisu:
	\begin{itemize}
		\item autoryzacja użytkowników,
		\item zarządzanie instancjami gier,
		\item korzystanie z serwisu za pomocą strony internetowej lub aplikacji na telefon,
		\item rozgrywka z botem. \\
	\end{itemize} 

Obecnie istnieją już takie rozwiązania, które dostarczają nam podobne funkcjonalności. Jednakże motywacją tej pracy jest bliższe zapoznanie się z procesem tworzenia takiego serwisu, jak również analiza złożoności gry Rummikub. Czynnikiem wyróżniającym się stworzonego serwisu jest umożliwienie graczom uczestniczenie w danej rozgrywce niezależnie od tego czy korzystają z aplikacji na telefon (iOS, Android), czy też ze strony internetowej w przeglądarce. \\

Praca składa się z czterech rozdziałów. W rozdziale pierwszym omówiono grę Rummikub, jej zasady oraz przebieg rozgrywki. W rozdziale drugim zobrazowano szczegółowy projekt serwisu. Do graficznego przedstawienia wykorzystano notacje UML. W rozdziale trzecim opisane zostały użyte technologie, które zostały użyte w projekcie informatycznym wraz z diagramami UML. W rozdziale czwartym przedstawiono dokumentację techniczną systemu. W rozdziale piątym przedstawiono kwestie implementacji bota w grze Rummikub.