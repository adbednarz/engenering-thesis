%Korekta ALD - nienumerowany wstęp
%\chapter{Wstęp}
\addcontentsline{toc}{chapter}{Wstęp}
\chapter*{Wstęp}

\thispagestyle{chapterBeginStyle}

Celem pracy jest zaprojektowanie i implementacja serwisu informatycznego, który umożliwia przeprowadzanie gier Rummikub.
\\

Główne założenia funkcjonalne serwisu:
	\begin{itemize}
		\item autoryzacja użytkowników,
		\item zarządzanie instancjami gier,
		\item korzystanie z serwisu za pomocą strony internetowej lub aplikacji na telefon,
		\item rozgrywka z botem. \\
	\end{itemize} 

Obecnie istnieją już takie rozwiązania, które dostarczają nam podobne funkcjonalności. Jednakże motywacją tej pracy jest bliższe zapoznanie się z procesem tworzenia takiego serwisu, jak również zaznajomienie się ze złożonością gry Rummikub. Wybrano nowe technologie do implementacji serwisu, dzięki czemu praca przyczyniła się również do poznania nieznanych dotąd technologii. Czynnikiem wyróżniającym się stworzonego serwisu jest umożliwienie graczom uczestniczenie w danej rozgrywce niezależnie od tego czy korzystają z aplikacji na telefon (iOS, Android), czy też ze strony internetowej w przeglądarce. \\

Praca składa się z czterech rozdziałów. W rozdziale pierwszym omówiono grę Rummikub, jej zasady oraz przebieg rozgrywki. W rozdziale drugim zobrazowano projekt serwisu. Opisano jego przypadki użycia oraz proces przebiegu aktywności.  W rozdziale trzecim przedstawiono technologie, które zostały użyte w projekcie informatycznym. Opisano narzędzia, z których korzystano podczas tworzenia serwisu. W rozdziale czwartym przedstawiono dokumentację techniczną systemu. Ukazano sposób w jaki zostały zaimplementowane serwer i aplikacja klienta. Do przedstawienia graficznego wykorzystano diagramy klas oraz sekwencji. W rozdziale piątym przedstawiono kwestie implementacji bota w grze Rummikub i umieszczono pseudokod algorytmu.