\chapter*{Podsumowanie}
\addcontentsline{toc}{chapter}{Podsumowanie}
\thispagestyle{chapterBeginStyle}

Obecnie serwis jest gotowy do wdrożenia. Wszystkie wskazane funkcjonalności zostały zaimplementowane. Serwis działa zarówno na stronie w przeglądarce jak i w postaci aplikacji mobilnej na telefon z systemem Android. Użytkownicy mogą zakładać własne konta i grać z innymi użytkownikami przez internet lub z botem bez potrzeby zakładania konta i łączności z internetem. \\

Główny fundament serwisu został stworzony, więc obecnie można rozszerzać serwis o nowe funkcjonalności. Najbardziej pożądaną kwestią jest dodanie bota po stronie serwera. Funkcjonalność ta przydałaby się w momencie, gdy chcemy zagrać w grę z większą liczbą osób, a niestety w pewnej chwili w serwisie nie ma zbyt wielu graczy. Ponadto należałoby rozwinąć możliwości związane z kontami użytkowników. Udostępniać statystyki dotyczące danego użytkownika, wprowadzić możliwość komunikacji tekstowej między graczami. Warto byłoby też pochylić się nad ulepszeniem wizualizacji aplikacji (dodaniem animacji czy efektów głosowych). Ważną kwestią jest również dalsza praca nad algorytmem zaawansowanego bota i uwzględnienie w nim możliwości wystąpienia dwóch dodatkowych kości z jokerami. \\

W projekcie tym udało się pomyślnie wykorzystać technologie Flutter oraz Firebase. Wymagało to przeznaczenia pewnej ilości czasu na ich naukę z uwagi na to, że są to dosyć młode technologie oraz serwis ten jest pierwszym większym projektem korzystającym z tych technologii dla autora. Z pewnością zapoznanie się z wieloplatformową technologią budowania aplikacji było cennym doświadczeniem. Wartością dodaną tego projektu jest szczególnie możliwość zapoznania się ze złożonością gry Rummikub.