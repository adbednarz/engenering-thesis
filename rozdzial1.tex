\chapter{Opis gry Rummikub}
\thispagestyle{chapterBeginStyle}
\label{rozdzial1}

\section{Gra}

Została stworzona przez Ephraima Hertzano i po raz pierwszy wydano ją w 1950 roku. W rozgrywce może uczestniczyć od dwóch do czterech graczy. Gra składa się z kości, które są numerowane od 1 do 13. Występują one w czterech kolorach (pomarańczowy, czerwony, niebieski, czarny). Każda kość o danym kolorze i liczbie występuje dwa razy. Ponadto występują dwie specjalne kości jako jokery. Łącznie gra zawiera 106 kości. \\

Gra polega na wykładaniu grup, bądź serii. Grupa to trzy lub cztery kości o różnych barwach, ale z tą samą liczbą, natomiast seria to co najmniej trzy kolejne kości o tym samym kolorze. \\

W przypadku braku odpowiednich kości do gry można posłużyć się dwoma zestawami kart po 52 karty i 2 jokery. Gdzie 1, 11, 12, 13 można zastąpić odpowiednio asem, waletem, damą i królem.

\section{Rozgrywka}

Zbiór wszystkich wymieszanych kości nazywany jest bankiem. Z niego na początku gry każdy gracz otrzymuje 14 kości. W przypadku pierwszego ruchu należy wyłożyć własne kości o sumie numerów tych kości co najmniej 30, bez możliwości modyfikowania kości leżących na planszy. \\ \\
Po wyłożeniu pierwszego ruchu gracz może modyfikować inne wyłożone wcześniej układy. Dozwolone jest przebudowywanie układów kości (rozbijanie lub rozbudowywanie). Jednak w każdym ruchu należy wyłożyć przynajmniej jedną własną kość. \\ \\
Na każdy ruch przypada ustalony wcześniej limit czasowy. Po jego upływie w przypadku braku wyłożenia kości przez gracza, lub gdy stan planszy nie spełnia zasad gry, gracz pobiera z banku jedną kość i cofa wprowadzone zmiany układów kości. \\ \\
Joker w grze Rummikub symbolizuje dowolną kość. Można nim zastąpić brakujący element do utworzenia układu kości. Joker ma taką samą wartość jak kość, którą zastępuje. Można zabrać wyłożonego jokera zastępując go odpowiednią kością i wykorzystać go w innym miejscu na planszy. \\ \\
Gra kończy się w momencie, gdy któryś z graczy wyłoży wszystkie swoje kości lub gdy zabraknie kości w banku. W przypadku drugim wygrywa ten gracz, który ma najmniejszą sumę liczb znajdujących się na kościach.


