\chapter{Bot}
\thispagestyle{chapterBeginStyle}

W ostatnim rozdziale omówiono aspekty związane z implementacją bota. Jakie podejścia zostały zrealizowane do zrealizowania tego celu oraz przeanalizowanie złożoności gry Rummikub. W związku z zastosowaniem architektury trójwarstwowej po stronie klienta instancja Bota zastępuję warstwę danych bez konieczności modyfikowania logiki aplikacji.

\subsection{Złożoność gry}

Gra Rummikub składa się z 106 kości, w których są dwa jokery. Joker może reprezentować sobą dowolną inną kość. Zatem 104 kości są numerowane od 1 do 13, w czterech różnych kolorach i każda kość występuje dwa razy. Kości można grupować w sekcje (rosnący ciąg liczb o tym samym kolorze) lub w grupy (kości o takiej samej liczbie z różnymi kolorami). Problemem jest znalezienie najlepszego wyłożenia podczas danego ruchu. \\



\subsection{Podstawowy}



\subsection{Zaawansowany}



%\cite{RummikubComplexity}