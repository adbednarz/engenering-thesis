\chapter{Implementacja serwisu}
\thispagestyle{chapterBeginStyle}

\section{Serwer}

Przy implementacji serwera wykorzystano głównie trzy moduły Firebase takie jak uwierzytelnianie, firestore (baza danych NoSQL) oraz funkcje. Jako metodę uwierzytelnienia użytkownika przy rejestracji wybrano opcję weryfikacji za pomocą adresu email.

\subsection{Baza danych}

W bazie danych czasu rzeczywistego Firestore zostały umieszczone dwie kolekcje - \emph{Users}, \emph{Games}. \\

W kolekcji \emph{Users} są przechowywane dokumenty indeksowane unikatowymi kodami ID użytkownika, które są przydzielane w czasie rejestracji. Każdy dokument zawiera pola \emph{name} i \emph{active}. Pole \emph{name} odnosi się do nazwy użytkownika, a pole \emph{active} jest wartością logiczną wskazującą, czy dany użytkownik jest dostępny w grze (jest zalogowany i obecnie nie znajduje się w rozgrywce z innymi graczami). Opcjonalne pole \emph{invitation} jest mapą zawierającą klucze \emph{gameId} oraz \emph{player}, których wartości wskazują kolejno na kod ID gry i nazwę gracza, który zaprosił danego gracza do swojej gry. \\

W kolekcji \emph{Games} są przechowywane dokumenty indeksowane automatycznie generowanymi unikatowymi kodami ID przy tworzeniu dokumentu i które są przypisane jako kody ID gier. Każdy dokument zawiera trzy pola: \emph{available} (wskazuje ile graczy może jeszcze dołączyć do gry), \emph{currentTurn} (wskazuje jakiego gracza jest teraz kolej), \emph{size} (wskazuje na liczbę graczy w grze). Ponadto dokument danej gry zawiera jeszcze subkolekcje \emph{playersQueue}, \emph{playersRacks}, \emph{pool}, \emph{state}. \\
Subkolekcja \emph{playersQueue} jest to zbiór dokumentów indeksowanych kodami ID graczy, którzy uczestniczą w tej grze. Każdy dokument zawiera pole \emph{name} z nazwą gracza oraz pole \emph{initialMeld}, które wskazuje czy dany gracz wyłożył już rozdanie początkowe. \\
Subkolekcja \emph{playersRack} to zbiór dokumentów z automatycznie generowanymi kodami ID. Dokumenty te przedstawiają kości, które są przydzielone graczowi. Każdy dokument zawiera dwa pola \emph{color} oraz \emph{number}. \\
Subkolekcja \emph{pool} to z kolei zbiór dokumentów z automatycznie generowanymi kodami ID. Zbiór tych dokumentów przedstawia bank w grze, czyli wszystkie kości, które nie znajdują się na planszy ani nie są przydzielone do gracza. Każdy dokument ma pola \emph{color} oraz \emph{number}. \\
Subkolekcja \emph{state} zawiera dokładnie jeden dokument \emph{sets} o najbardziej złożonej strukturze. W dokumencie \emph{sets} znajdują się wszystkie zbiory, które są wyłożone na planszy. Struktura tego dokumentu składa się z mapy, w której klucze to pozycja pierwszej kości z danego zbioru na planszy, a więc moment, w którym rozpoczyna się dany zbiór kości. Wartości tej mapy to tablica kości, gdzie każda kość jest w postaci mapy z kluczami \emph{color} i \emph{number}. Taki sposób przedstawienia stanu planszy wynika głównie z optymizacji kosztów modyfikowania bazy danych. Nie rozbito każdego zbioru kości na osobne dokumenty, ponieważ zwiększa to nam liczbę zliczanych zapisów do bazy danych. Ponadto w tym przypadku nie ma potrzeby korzystania z właściwości oferowanych przez dokumenty takie jak śledzenie zmian w danym dokumencie czy możliwość tworzenia prostych zapytań SQL na zbiorze dokumentów.


\subsection{Funkcje}

Za pomocą funkcji Firebase został zaimplementowany kod serwera naszego serwisu. Jego głównymi zadaniami jest zarządzanie instancjami gier oraz uniemożliwienie prób oszukiwania w czasie rozgrywki przez graczy za pomocą ingerencji w kod źródłowy aplikacji po stronie klienta. \\

Zbiór zdefiniowanych funkcji został uporządkowany w trzy podzbiory. W pierwszym podzbiorze znajdują się wszystkie możliwe do wywołania przez użytkownika funkcje serwera. Drugim podzbiorem jest klasa statycznych funkcji odpowiedzialnych za zarządzanie instancjami gier. Trzecim podzbiorem jest klasa statycznych funkcji odpowiadające za walidację ruchów użytkowników w grze i implementację logiki rozgrywki gry. \\



\section{Klient}

Warstwa klienta w serwisie napisana we Flutterze jest dostępna jako aplikacja na telefon albo jako strona internetowa. Z uwagi na wieloplatformowość Fluttera kod źródłowy obu wersji klienta jest wspólny z odpowiednimi konfiguracjami i modyfikacjami na poszczególne platformy.

\subsection{Diagramy klas}

\subsection{Diagramy sekwencji}


